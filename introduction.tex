\chapterimage{puyobg.jpg} % Chapter heading image
\chapterspaceabove{6.75cm} % Whitespace from the top of the page to the chapter title on chapter pages
\chapterspacebelow{7.25cm} % Amount of vertical whitespace from the top margin to the start of the text on chapter pages

\chapter{Introduction}

\section{What is Puyo Puyo?}

If you've never played Puyo Puyo before, you might get the impression that it is a quirky ``version'' of another popular puzzle game, such as Tetris or Candy Crush. Be careful, though -- many Puyo players will take offense if you talk about it this way!

In fact, the goal of this section is to argue for the opposite: that Puyo Puyo is a \emph{very} unique game, and that beneath the surface lie many complexities which serve only to enrich the game and make it all the more interesting. Hopefully, this will encourage you to take a deeper dive, and explore Puyo Puyo for yourself.

\begin{note}
    From now on, we'll refer to Puyo Puyo as just Puyo. There's no point in repeating ourselves, right?
\end{note}

\subsection{Basic elements}

There are two elements that are indispensable for any Puyo game. The first is the \textbf{board}\index{Board}, which is just a grid where you can place puyos. The second is, well, the \textbf{puyos}! Here is an example of a board with a few puyos.

\begin{figure}[h]
    \centering
    \begin{puyotikz}[\puyobigscale]
        \centering
        \puyoboard[nrows=6, ncols=6, nhidrows=0]{rggpg/rgrp/brbp/bbp/pp/p}{pr/gg}
    \end{puyotikz}
\end{figure}

We have a few things to unpack here, so bear with me. As you can see, this is a $6 \times 6$ board. You can fit one puyo onto each square. In most official game modes, you'll receive a $6 \times 13$ board to work with; however, since we're just getting started, a simpler board will suffice.

The round shapes within the board are our puyos. Notice that puyos of the same color connect. The purples in row 4 are an example of that. This connective property will become very important soon, so keep it in mind!

You've probably also noticed that there are two pairs (pieces) of puyos outside of the board. This is what we call a \textbf{piece preview}\index{Piece preview} -- it lets us know which pieces we'll receive next, and lets us improve our game plan. Typically, we receive the piece on top first, and then the piece at the bottom. There's no limit to the number of pieces we can receive -- it all depends on how long the game lasts!

\begin{terminology}
    In the example above, the second piece in the preview is made of two green puyos. I'll be referring to pieces like this as ``\textbf{double greens/reds/blues/etc.}\index{Double piece}''
\end{terminology}

\begin{note}
    The letters and numbers in the example above are \textbf{not} present in most versions of the game. We'll make use of them in this compendium to make your life -- and mine -- easier. For instance, instead of saying ``the group of purples that look like an L on the right side of the board,'' we can just say ``the purples at \texttt{2e}.''
\end{note}

\subsection{Chains}

In the previous section, I mentioned that puyos of the same color connect. There is a reason for this: when you connect four puyos of the same color, they pop! This is also true if you manage to make groups with five or more puyos.

This property of groups is crucial, since it allows us to build \textbf{chains}\index{Chain}. If you know a bit of chemistry, you're probably familiar with the concept of \emph{chain reactions}. Well, Puyo chains are very similar: if you pop the right group of puyos, others will follow suit, and soon you'll have popped dozens of puyos indirectly!

Consider the following example. Here, we have 2 groups. There are the reds in row 1, the greens in row 2, and a lonely red in row 3. Our piece preview is giving us double reds and greens. What do we do here?

\begin{figure}[h]
    \centering
    \begin{puyotikz}[\puyobigscale]
        \centering
        \puyoboard[nrows=6, ncols=6, nhidrows=0]{rgr/rg/rg/}{rr/gg}
    \end{puyotikz}
\end{figure}

Let's get the most obvious option out of the way. What happens if we connect our reds to the ones in row 1? One way to do it is as follows:

\begin{figure}[H]
    \centering
    \href{https://www.puyo.gg/simulator/chain/mjrmctsv4r}{
        \begin{puyotikz}[\puyobigscale]
            \centering
            \puyoboard[nrows=6, ncols=6, nhidrows=0]{rgr/rg/rg/rr}{gg}
        \end{puyotikz}
    }
    \begin{puyotikz}[\puyobigscale]
        \centering
        \puyoboard[nrows=6, ncols=6, nhidrows=0]{gr/g/g}{gg}
    \end{puyotikz}
\end{figure}

After we connect the 5 reds, the group pops. Because Puyo follows the laws of gravity, everything above this group falls to the bottom of the board, and we are left with the board state on the right. Very cool! So what?

Well, actually, this is quite boring, right? There's nothing special about it. Let's take an alternative route. We'll begin by placing the reds on \texttt{4a}, and then we'll pop the greens instead, like this:

\begin{figure}[H]
    \centering
    \begin{puyotikz}[\puyobigscale]
        \centering
        \puyoboard[nrows=6, ncols=6, nhidrows=0]{rgrrr/rg/rg}{gg}
    \end{puyotikz}
    \href{https://www.puyo.gg/simulator/chain/mjrofoqv2y}{
        \begin{puyotikz}[\puyobigscale]
            \centering
            \puyoboard[nrows=6, ncols=6, nhidrows=0]{rgrrr/rg/rggg}{}
        \end{puyotikz}
    }
\end{figure}

Once the greens pop, we get this:

\begin{figure}[H]
    \centering
    \begin{puyotikz}[\puyobigscale]
        \centering
        \puyoboard[nrows=6, ncols=6, nhidrows=0]{rrrr/r/r}{}
    \end{puyotikz}
    \begin{puyotikz}[\puyobigscale]
        \centering
        \puyoboard[nrows=6, ncols=6, nhidrows=0]{}{}
    \end{puyotikz}
\end{figure}

Great! With just one piece, we got rid of \emph{two} groups! On top of that, our board is now empty -- which, as we'll see, gives us an interesting advantage in competitive Puyo!

\begin{terminology}
    An \textbf{All Clear}\index{All Clear} (AC) happens when you pop all the puyos on your board, leaving it empty. Note that the empty board state at the start of a match doesn't count as an all clear!
\end{terminology}

\begin{terminology}
    The \textbf{trigger}\index{Trigger} is the group of puyos you pop in order to start a chain. In this example, it's the greens.
\end{terminology}

At this point, you are saying one of two things:

\begin{itemize}
    \item ``I think I need a break, this is already too complicated for me.''
    \item ``This is a game for babies! Do people really need a tutorial for this?''
\end{itemize}

If you're the first type of person, do take a break! Funnily enough, gravity is the first major roadblock you'll encounter in Puyo. It's perhaps the hardest thing to wrap your mind around, and even veterans make mistakes every now and then, which can mostly be traced back to gravity. Once your mind is at ease again, you can come back.

Here is an exercise. Try to understand how these chains work:

\begin{figure}[H]
    \centering
    \href{https://www.puyo.gg/simulator/chain/mjrmz0lh44}{
        \begin{puyotikz}[\puyobigscale]
            \centering
            \puyoboard[nrows=6, ncols=3, nhidrows=0]{rgbgr/rgbb/rgb}{}
        \end{puyotikz}
    }
    \href{https://www.puyo.gg/simulator/chain/mjrn02i161}{
        \begin{puyotikz}[\puyobigscale]
            \centering
            \puyoboard[nrows=6, ncols=3, nhidrows=0]{rgbyyb/rgbyg/rgbyr}{}
        \end{puyotikz}
    }
    \href{https://www.puyo.gg/simulator/chain/mjrodeqh4s}{
        \begin{puyotikz}[\puyobigscale]
            \centering
            \puyoboard[nrows=6, ncols=4, nhidrows=0]{rrr/bbbr/gggb/g}{}
        \end{puyotikz}
    }
    \href{https://www.puyo.gg/simulator/chain/mjroeaso2h}{
        \begin{puyotikz}[\puyobigscale]
            \centering
            \puyoboard[nrows=6, ncols=4, nhidrows=0]{pprpg/pgrrr/ggpy/yyyppp}{}
        \end{puyotikz}
    }
\end{figure}

\begin{note}
    If you click on one of the boards above, you'll be sent to \href{https://www.puyo.gg/simulator/}{\textbf{puyo.gg}}, a free simulator that allows you to build theoretical chains and practice. Another popular chainsim is \href{https://puyonexus.com/chainsim/}{\textbf{Puyo Nexus' chainsim}}. Pay attention to the board coordinates -- if they are orange, that means the board is clickable!
\end{note}

The first board is similar to what we've seen so far. The groups pop in the order $B \rightarrow G \rightarrow R$ -- blues, then greens, then reds; keep this notation in mind! This gives us a bigger chain than before, but it's now harder to build and read.

The second board is more interesting. We add the yellows to our stack of groups, and the pop order is now $Y \rightarrow B \rightarrow G \rightarrow R$. Note that it's perfectly fine to put the green and the red on \texttt{5b} and \texttt{5c}. In fact, we can swap the positions of every puyo on top of the yellows, as long as the yellows still pop.

The third board is an example of what we call \textbf{stairs}\index{Stairs}. Stairs consists of multiple pillars of puyos, stacked side by side, with the puyo that connects to the next group being placed on top of the previous group. Stairs is one of the easiest patterns to learn, which is why you'll often see beginners using it. The pop order here is $G \rightarrow B \rightarrow R$.

The fourth board is the weirdest of them all. I urge you to click on it, if you're having trouble visualizing it. The pop order is $R \rightarrow P \rightarrow G \rightarrow Y \rightarrow P$.

\begin{remember}
    Hubris is one of your greatest enemies. Sometimes, you just have to watch a chain in action in order to understand what's going on. This is why watching Puyo matches is just as important as playing the game.
\end{remember}

I bet the game doesn't seem so simple now, does it? But don't think about those chains too much. For now, I want you to keep in mind that Puyo is all about gravity.

Another thing I'd like to bring up is the fact that our vocabulary is quite limited right now. Let's remedy that. Other than weird shapes, you've probably noticed that the boards differ in terms of \textbf{chain length}. The chain length tells us how many groups have to be popped in sequence for a chain to \textbf{resolve} (finish). For example, the first board has pop order $B \rightarrow G \rightarrow R$. That's a sequence of 3 groups, so we have a \textbf{3-chain}. The other boards contain 4-, 3-, and 5-chains, respectively.

Notice how I've been talking about \emph{sequences}. By that, I mean that two groups popping \emph{at the same time} constitute \emph{a singular part of the chain}. For example, the following chain is only a 2-chain, not a 3-chain.

\begin{figure}[H]
    \centering
    \href{https://www.puyo.gg/simulator/chain/mjrp5ngl1v}{
        \begin{puyotikz}[\puyobigscale]
            \puyoboard[nrows=3, ncols=6, nhidrows=0]{bb/b/bgr/gr/gr/gr}{}
        \end{puyotikz}
    }
\end{figure}

\begin{terminology}
    We use the term \textbf{link}\index{Link} to refer to each part of a chain sequence. In the example above, the blues constitute the first link, while the reds and greens together constitute the second link.
\end{terminology}

\begin{terminology}
    \textbf{Power chain}\index{Power chain} is a loose term used to describe chains where multiple groups pop in a single link. It can also be used for chains with really big groups.
\end{terminology}

\begin{terminology}
    \textbf{Power bonus}\index{Power bonus} is the extra damage a chain sends for having extra puyos or groups per link.
\end{terminology}

So why am I talking about chain lengths? That's what we'll see next.

\subsection{Game objective}

The game's objective depends on what you are playing. If you are playing practice mode, or any other single-player mode (such as the score and time challenges from Puyo Puyo Champions), you'll focus on developing a certain type of efficiency, or a self-imposed goal. There are certain skills that you might find easier to develop in single-player, since it puts you under no pressure from other players, allowing you to think more clearly.

I'll focus on \textbf{versus} here. By that, I mean any adversarial mode: whether you're playing against a CPU or another human, the same objective applies.

So, what is your objective in versus mode? Naturally, it's to defeat your opponent. To do that, you need your opponent to \textbf{top out}. A top out occurs when a player fills their third column completely. When that happens, the game can't spawn any pieces on the player's board, and they lose. In most games, you'll see a red X on the topmost slot of the third column. You don't want to put a piece there.

\begin{figure}[H]
    \centering
    \begin{puyotikz}[\puyobigscale]
        \puyoboard[nrows=12, ncols=6, nhidrows=0]{//bgrrgybgyyrb///}{}
    \end{puyotikz}
\end{figure}

Of course, if your opponent also knows the objective, they won't top out by themselves. So how do you make them top out?

\subsection{Sending damage}

Your chains cause damage to the opponent. What's more, the lengthier the chain, the stronger it will be. That's why I've been telling you about that. But how exactly do chains ``damage'' your opponent? Well, if the idea is to make them top out, then you fill their board with garbage!

\begin{figure}[H]
    \centering
    \begin{puyotikz}[\puyobigscale]
        \puyoboard[nrows=12, ncols=6, nhidrows=0]{nnnnn/nnnnn/nnnnn/nnnnn/nnnnn/nnnnn}{}
    \end{puyotikz}
\end{figure}

The ``gray puyos'' on this board are actually \textbf{garbage}\index{Garbage}, or \textbf{nuisance} puyos. These puyos have a few quirky properties that make them quite annoying to deal with. For one, they don't connect with each other, so the garbage in the example above won't pop by itself. To make garbage pop, you have to pop a group of puyos that's \emph{adjacent to it}. Consider this example.

\begin{figure}[H]
    \centering
    \href{https://www.puyo.gg/simulator/chain/mjrtn2bt1u}{
        \begin{puyotikz}[\puyobigscale]
        \puyoboard[nrows=12, ncols=6, nhidrows=0]{nnnnnr/nnnnnr/nnnnnr/nnnnnr/nnnnn/nnnnn}{}
        \end{puyotikz}
    }
    \begin{puyotikz}[\puyobigscale]
        \puyoboard[nrows=12, ncols=6, nhidrows=0]{nnnn/nnnn/nnnn/nnnn/nnnnn/nnnnn}{}
    \end{puyotikz}
\end{figure}

Another property of garbage\index{Garbage} is that it doesn't fall in pairs, like the pieces you receive. Instead, garbage is sent in bulk. This can create very uncomfortable situations. For example, let's suppose your opponent is building something big, like this.

\begin{figure}[H]
    \centering
    \href{https://www.puyo.gg/simulator/chain/mjrtzhrc3n}{
        \begin{puyotikz}[\puyobigscale]
        \puyoboard[nrows=12, ncols=6, nhidrows=0]{rrr/gggr/bbbg/yyyb/rrryb/gggbbgr}{}
        \end{puyotikz}
    }
\end{figure}

This is a 7-chain, and that's huge. As we'll see in a moment, it's enough to instantly defeat an opponent. Now, if you manage to make a chain that sends 12 garbage, for example, \emph{all 12 garbage will be sent at once}, like this:

\begin{figure}[H]
    \centering
    \href{https://www.puyo.gg/simulator/chain/mjru5ax96k}{
        \begin{puyotikz}[\puyobigscale]
        \puyoboard[nrows=12, ncols=6, nhidrows=0]{rrrnn/gggrnn/bbbgnn/yyybnn/rrrybnn/gggbbgrnn}{}
        \end{puyotikz}
    }
\end{figure}

Now your opponent can't access their chain, and they'll have to \textbf{dig}\index{Digging} through the garbage you sent to make their 7-chain work. I recommend trying out this problem for yourself, if you're a beginner. Click on the board above and try to dig through the garbage. See how long it takes you to send the 7-chain! (Tip: You have to place a blue puyo in column 5.)

As you can see, I kind of lied to you. There is an objective in Puyo that precedes beating your opponent, and that is to survive. In the situation above, we aren't really trying to make our opponent top out yet. Instead, we are protecting ourselves by sending enough garbage to our opponent's board that they won't be able to use their chain anymore, which could make \emph{us} top out! Trust me, though, we're only scratching the surface here. And that's perfectly fine when you're a beginner!

\subsubsection{Score system}

Let's talk about how Puyo's garbage system works! We must first look into the game's score system\index{Score}. The score is the number you'll see underneath your board in the games. It tells you how much garbage\index{Garbage} your chain is going to send. The score is determined by \href{https://puyonexus.com/wiki/Scoring}{a pretty complicated formula} that we aren't really interested in. What really matters is that there are four variables involved:

\begin{itemize}
    \item How many puyos were cleared in the chain.
    \item How long your chain is.
    \item How many puyos popped in each group.
    \item How many groups of different colors popped at the same time.
\end{itemize}

Once we delve into sophisticated versus tactics, each of these items will become very important. Because all of these things are taken into account, Puyo is a very situational game. For example, depending on the situation, a chain that pops a lot of groups in one link is better than a lengthy chain. This depends entirely on context, though.

\begin{terminology}
    Dropping the pieces faster (usually by pressing down) is known as \textbf{soft dropping}\index{Soft drop}. You'll notice that soft dropping also increases your score, which is important in very specific circumstances.
\end{terminology}

To determine how many garbage puyos the opponent will receive, the game divides your chain's score by the \emph{target points}\index{Target points}, a hidden variable that's 70 by default. To send 2 lines of garbage like in the previous example, you would need a score of 840, which lies somewhere between a 2-chain and a 3-chain. This information isn't super useful in the heat of a match, but it becomes relevant when discussing the difference in power between two big chains.

\begin{note}
    This is a slightly simplified view of the garbage calculation. As explained in the article linked above, the remainder of the division by the target points is stored and included in future calculations.
\end{note}

\begin{terminology}
    The target points decrease over time, which means your chains send more damage the longer a match lasts. The \textbf{margin time}\index{Margin time} is a variable you can configure before a match, and it defines how quickly the target points decrease.
\end{terminology}

\begin{remember}
    Knowing the numbers will hardly ever help you in Puyo. Intuition is orders of magnitude more important than the math behind the game! To develop your intuition, you'll have to play and see for yourself how strong each attack is.
\end{remember}

\subsubsection{Garbage in batches}

As I said before, garbage\index{Garbage} is sent in bulk to the opponent. Up to 30 garbage puyos can be sent to the opponent at once, which is equivalent to 5 rows of garbage. When you don't send enough garbage to complete a line, the garbage is sent randomly at the incomplete line. For example, if you send 14 garbage puyos, the two rows at the bottom will be completely filled, but the row on top may fall in multiple different ways.

\begin{figure}[H]
    \centering
    \begin{puyotikz}[\puyobigscale]
        \puyoboard[nrows=3, ncols=6, nhidrows=0]{nn/nn/nnn/nn/nn/nnn}{}
    \end{puyotikz}
    \begin{puyotikz}[\puyobigscale]
        \puyoboard[nrows=3, ncols=6, nhidrows=0]{nnn/nn/nn/nn/nnn/nn}{}
    \end{puyotikz}
\end{figure}

\begin{terminology}
    \textbf{One in six}\index{One in six} is a term you'll hear often in competitive Puyo. It's mostly used in situations where one player sends a single garbage puyo to the opponent's board, and it lands on a spot that breaks the opponent's chain, or makes it inaccessible, or even makes them top out.
\end{terminology}

Puyo has its own quirky system of symbols to indicate how much garbage your opponent is going to receive. When you send a chain, the symbols start appearing over your opponent's board. This is known as the \textbf{garbage queue}\index{Garbage!Queue}.

\begin{table}[H]
    \centering
    \begin{tabular}{llc}
        \textbf{Symbol} & \textbf{Name} & \textbf{Amount of garbage}\\
        \hline
        \icon{pebble} & Pebble & 1 \\
        \icon{line} & Line & 6 \\
        \icon{rock} & Red Rock & 30 \\
        \icon{star} & Star & 180 \\
        \icon{moon} & Moon & 360 \\
        \icon{crown} & Crown & 720\\
        \hline
    \end{tabular}
\end{table}

As an example, if you send \icon{rock}\icon{line}\icon{line} garbage to your opponent, that's $30 + 6 + 6 = 42$ garbage puyos. Once again, the precise math hardly ever matters in a match. You should know that, most of the time, you're in big trouble if you receive a red rock; and you're most likely dead if you receive two.

\subsubsection{All Clear bonus}

Recall that an AC\index{All Clear} happens when you send a chain that empties your board. This triggers a special mechanic that adds a red rock to the damage output of your next chain. This makes ACs especially relevant early in a match, and it can give a big advantage to whoever manages to find an AC first.

\begin{terminology}
    An \textbf{All Clear start}\index{All Clear!Start} happens when there is a trivial AC at the start of a match. For example, when a match starts with two double reds on the piece preview.
\end{terminology}

\begin{terminology}
    \textbf{All Clear fishing}\index{All Clear!Fishing} describes the act of building chains in unusual ways in an attempt to find an AC. This is a risky strategy at higher levels of play.
\end{terminology}

\subsection{Fighting back}

So far, I've been talking about this game as if you were the only one who can send damage to your opponents -- never the other way around. So here's an important question: what if you're the one under attack? Is there anything you can do to defend yourself?

The oldest games in the Puyo series \emph{didn't give you any way to fight back}. The star, moon and crown symbols didn't even exist back then, because they would've been useless. Thankfully, it wasn't long before the \textbf{offset system}\index{Offset} was invented. Its premise is simple: in order to counter the opponent's attack, you have to send a stronger attack!

Let's look at an example. Suppose you're playing versus, and your opponent (right side) starts sending their chain.

\begin{figure}[H]
    \centering
    \begin{puyotikz}[\puyobigscale]
        \puyoboard[nrows=12, ncols=6, nhidrows=0]{rg/rg/rg/br/b/byy}{gy/yb}
    \end{puyotikz}
    \href{https://www.puyo.gg/simulator/chain/mjsrbjim10}{
        \begin{puyotikz}[\puyobigscale]
            \puyoboard[nrows=12, ncols=6, nhidrows=0]{rrr/bbbr/gggb/g/y/y}{gy/yb}
        \end{puyotikz}
    }
\end{figure}

The opponent has a 3-chain, and it'll send \icon{line}\icon{line}\icon{pebble}\icon{pebble} (2 lines and 2 randomly placed pebbles). What do you do here?

The first thing to keep in mind is that no garbage will fall on your board until the following sequence of events happens:

\begin{enumerate}
    \item The opponent's chain resolves.
    \item You place a piece on the board.
\end{enumerate}

So, you're actually safe for a few seconds. And luckily, you have many response options here! Here's what many desperate beginners would do:

\begin{enumerate}
    \item Notice the incoming chain.
    \item Panic!
    \item Notice that they can send a 2-chain with the blues ($B \rightarrow R$).
    \item Throw the first piece on the preview to the right.
    \item Send the 2-chain with the blues.
\end{enumerate}

\begin{figure}[H]
    \centering
    \href{https://www.puyo.gg/simulator/chain/mjss4war5h}{
        \begin{puyotikz}[\puyobigscale]
            \puyoboard[nrows=12, ncols=6, nhidrows=0]{rg/rg/rg/br/bby/byygy}{}
        \end{puyotikz}
    }
\end{figure}

The result is the most basic 2-chain, totalling \icon{pebble}\icon{pebble}\icon{pebble}\icon{pebble}\icon{pebble} (5) garbage puyos. Let's do the math: the opponent sent 14 garbage to your queue, and you responded with 5 garbage. That means you take $14 - 5 = 9$ garbage of damage instead. This is how you use the offset system to \emph{reduce the damage you take}.

Let's explore three different scenarios now. Can you figure out how to arrive at these board states using the piece preview? Which one do you think is better?

\begin{figure}[H]
    \centering
    \href{https://www.puyo.gg/simulator/chain/mjsst1c836}{
        \begin{puyotikz}[\puyobigscale]
            \puyoboard[nrows=12, ncols=6, nhidrows=0]{rg/rg/rg/br/bgy/byyyb}{}
        \end{puyotikz}
    }
    \href{https://www.puyo.gg/simulator/chain/mjsstwa26d}{
        \begin{puyotikz}[\puyobigscale]
            \puyoboard[nrows=12, ncols=6, nhidrows=0]{rg/rg/rg/brgy/bby/byy}{}
        \end{puyotikz}
    }
    \href{https://www.puyo.gg/simulator/chain/mjssujzwx}{
        \begin{puyotikz}[\puyobigscale]
            \puyoboard[nrows=12, ncols=6, nhidrows=0]{rg/rg/rg/brg/byyb/byy}{}
        \end{puyotikz}
    }
\end{figure}

All of these are 3-chains. The first one has pop order $Y \rightarrow B \rightarrow R$. Every link in this chain matches every link in the opponent's chain: only one group pops per link, and every group has 4 puyos. Hence, this chain also sends \icon{line}\icon{line}\icon{pebble}\icon{pebble} garbage. As a result, you receive no garbage! This is known as a \textbf{perfect counter}\index{Perfect counter}.

\begin{note}
    Perfect counters are so rare that they are almost guaranteed to surprise players and spectators alike. However, they usually give a slight advantage to the opponent, since they have time to build a new chain while your chain is offsetting theirs.
\end{note}

The chain in the middle has pop order $B \rightarrow (G, R) \rightarrow Y$. This chain matches the opponent's in chain length and number of puyos popped in each group; however, it pops two groups in the second link. This makes it a stronger chain, and sends \icon{line}\icon{line}\icon{line}\icon{pebble}\icon{pebble}\icon{pebble} (21 garbage). Since this is more garbage than what the opponent sent, you end up \emph{countering} the opponent, and sending $21 - 14 = 7$ garbage their way.

How about the chain on the right? This one has pop order $Y \rightarrow B \rightarrow (G, R)$. As you might expect, this chain is also stronger than the opponent's, for the same reason as before. However, it is also stronger than the chain in the middle, because the reds and greens pop in the third link. This chain sends a whopping \icon{line}\icon{line}\icon{line}\icon{line}\icon{pebble}\icon{pebble} (26 garbage), totalling 2 lines of garbage against the opponent.

\begin{remember}
    Power chains tend to be stronger when the power bonus happens closer to the end of the chain.
\end{remember}

Notice that the last 2 chains result in ACs, so you'll send 5 extra rows of garbage the next time you send a chain. So not only are they stronger than the perfect counter -- they are \emph{much} stronger. But that doesn't mean the game is over for your opponent! While you're busy offsetting their chain, they have time to build a new chain, which gives them the chance to \emph{counter your counter}! This back-and-forth is called a \textbf{volley}\index{Volley}. Volleys last longer when both players build big chains. That's because longer chains give players more time to prepare a counter. There's \href{https://youtu.be/UXNnO20vy-c?si=Gf1ifRKK8fuzeNLi}{a popular TAS\footnote{TAS stands for Tool Assisted Speedrun, and it's often used to refer to video game footage where the player uses specialized software to access the game's memory and input commands frame by frame, resulting in ``perfect'' gameplay.} video} that showcases an 8-minute match full of volleys!

\subsection{Random number generator}

By now, I believe it's become clear just how different Puyo is from other puzzle games, and why some even go as far as to call it a fighting game. For all intents and purposes, there is an infinite number of single-player games. Versus mode creates even more possibilities. Chains are akin to combos in a fighting game, and responding to attacks requires great vision and understanding of various mechanics and strategies that we have yet to explore.

But I would argue that Puyo's most defining feature is its random number generator (RNG)\index{Random number generator}. RNG exists in most modern games, and it's basically an algorithm that allows random\footnote{RNG in games is actually \emph{pseudo-random}. Usually, the computer takes a variable (for example, the current time) and applies a bunch of formulas to it. The result is a seemingly random number.} events to occur. In Puyo, the RNG is responsible for generating the pieces that you receive. This is a game changer, because it introduces unpredictability to the game.

Let's explore a simple example. I showed you plenty of horizontal stacks in previous sections. So let's say you're very excited to practice. You close this compendium and open a Puyo game. You can't wait to build a stack like this.

\begin{figure}[H]
    \centering
    \begin{puyotikz}[\puyobigscale]
        \puyoboard[nrows=5, ncols=6, nhidrows=0]{ybpby/ybpp/ybp}{}
    \end{puyotikz}
\end{figure}

Ideally, you'd want a piece preview like this: $YB \rightarrow YB \rightarrow YB \rightarrow PB \rightarrow PP \rightarrow PY$. This is highly unlikely! Let's look at a few roadblocks that might emerge as you try to build this chain.

\subsubsection{Roadblock 1: Unwanted pieces}

This one is really common. In our ideal world, we only use 3 different colors for our chain. However, in standard games, we have 4 colors to work with. So situations like this are very common.

\begin{figure}[H]
    \centering
    \begin{puyotikz}[\puyobigscale]
        \puyoboard[nrows=5, ncols=6, nhidrows=0]{ybpb/yb}{rr/ry}
    \end{puyotikz}
\end{figure}

What do you do with those reds that showed up all of a sudden? What most beginners do is throw \textbf{unwanted pieces}\index{Unwanted piece} to the side -- often column 6. This is completely fine when you're still learning how to chain, but it becomes a harmful habit after you reach a certain level of competence. Here's an example of what a beginner would likely do (left), versus an experienced player (right).

\begin{figure}[H]
    \centering
    \begin{puyotikz}[\puyobigscale]
        \puyoboard[nrows=5, ncols=6, nhidrows=0]{ybpb/yb////rryr}{}
    \end{puyotikz}
    \begin{puyotikz}[\puyobigscale]
        \puyoboard[nrows=5, ncols=6, nhidrows=0]{ybpb/yb/y/r/r/r}{}
    \end{puyotikz}
\end{figure}

The chain on the right is better for a few reasons: (a) it completes the group of yellows; (b) it creates a group of reds that's ready to pop; (c) the reds can be added as a fourth link in the chain (as shown below); (d) it makes the board more even in terms of height (a very important thing that we will discuss later).

\begin{figure}[H]
    \centering
    \href{https://www.puyo.gg/simulator/chain/mjsxs4kk3w}{
        \begin{puyotikz}[\puyobigscale]
            \puyoboard[nrows=5, ncols=6, nhidrows=0]{ybpby/ybpp/ybpr/r/r/r}{}
        \end{puyotikz}
    }
\end{figure}

\begin{remember}
    Most unwanted pieces are only unwanted in the short term. The most efficient players think many steps ahead when placing seemingly useless pieces.
\end{remember}

\subsubsection{Roadblock 2: Drought}

A \textbf{drought}\index{Drought} occurs when one or more colors take too long to appear in the preview. Droughts can be extremely painful when you need a certain color to complete your chain. Let's take advantage of the previous example to elaborate on this. Suppose you're in this pickle when trying to build your ideal chain.

\begin{figure}[H]
    \centering
    \begin{puyotikz}[\puyobigscale]
        \puyoboard[nrows=6, ncols=6, nhidrows=0]{ybpb/yb/y/r/r/r}{yb/rb}
    \end{puyotikz}
\end{figure}

No purples in sight! This chain is quickly becoming unsustainable. Let's first look at 3 possibilities on how to proceed.

\begin{figure}[H]
    \centering
    \begin{puyotikz}[\puyobigscale]
        \puyoboard[nrows=6, ncols=6, nhidrows=0]{ybpb/yb/y/r/r/rybrb}{}
    \end{puyotikz}
    \begin{puyotikz}[\puyobigscale]
        \puyoboard[nrows=6, ncols=6, nhidrows=0]{ybpbby/yb/ybr/r/r/r}{}
    \end{puyotikz}
    \begin{puyotikz}[\puyobigscale]
        \puyoboard[nrows=6, ncols=6, nhidrows=0]{ybpbby/yb/yr/rb/r/r}{}
    \end{puyotikz}
\end{figure}

Now let's discuss each one of these, from left to right.

\begin{enumerate}
    \item This, again, is panic-mode beginner stuff. You're not really looking for a solution here, you're just trying to build exactly what you have in mind. This is bad, and if the drought worsens, so does your chain!
    \item A more level-headed player's approach. You acknowledge that it's okay for the chain to be a bit different from what you envisioned, hence the blue-yellow on \texttt{5a}. You even make use of the reds! This is much better than option 1, but now the purples are in a bit of an awkward spot, since column 3 is taken by a red puyo.
    \item A very classy option! Just like the previous chain, you're making sure the yellows will pop. Plus, the red-blue was placed in a way that allows you to extend the purples as envisioned, with the reds still connecting at the end of the chain. This is the best option of the bunch.
\end{enumerate}

Now I present an unintuitive possibility to you.

\begin{figure}[H]
    \centering
    \begin{puyotikz}[\puyobigscale]
        \puyoboard[nrows=6, ncols=6, nhidrows=0]{ybpb/yb/yb/ryb/rr/r}{bb/yb}
    \end{puyotikz}
\end{figure}

This one is a big brain move, and it's actually better than number 3 above depending on the situation. Take a look at the piece preview. There's still no purple coming. Now, your opponent probably depends on purples to make their chain work, as well. So, while they wait to receive the purples, you build a \textbf{secondary trigger}\index{Secondary trigger} with the reds, which results in a power 2-chain. This sends \icon{line}\icon{line}\icon{pebble} to your opponent! Two lines of garbage plus a drought is a huge hit, and it might determine your opponent's fate if you keep playing well.

\begin{note}
    I would not recommend building this attack if, instead of $BB \rightarrow BY$, the piece preview contained purples. You'd be risking a counter from your opponent.
\end{note}

Lastly, I'd like to bring this possibility up.

\begin{figure}[H]
    \centering
    \begin{puyotikz}[\puyobigscale]
        \puyoboard[nrows=6, ncols=6, nhidrows=0]{ybpbby/yb/yb/rr/r/r}{}
    \end{puyotikz}
\end{figure}

This one is unintuitive because many players (not just beginners!) have a hard time realizing that it's okay to pop a few groups here and there to make your life easier. The world won't end if you just get rid of the reds here! While I don't think this is the best option, it might make your life easier, so you can go for it.

\subsubsection{Roadblock 3: Bad piece order}

Sometimes you receive exactly the pieces you need to build a chain, but you have a hard time figuring out how to build it, because you had a different build order in mind. At the start of this section, I mentioned that you'd probably want this sequence of pieces in the piece preview: $YB \rightarrow YB \rightarrow YB \rightarrow PB \rightarrow PP \rightarrow PY$. That's because I'm assuming most people would find it easier to build this chain like this.

\begin{figure}[H]
    \centering
    \begin{puyotikz}[\puyobigscale]
        \puyoboard[nrows=6, ncols=3, nhidrows=0]{yb/////}{}
    \end{puyotikz}
    \begin{puyotikz}[\puyobigscale]
        \puyoboard[nrows=6, ncols=3, nhidrows=0]{yb/yb////}{}
    \end{puyotikz}
    \begin{puyotikz}[\puyobigscale]
        \puyoboard[nrows=6, ncols=3, nhidrows=0]{yb/yb/yb///}{}
    \end{puyotikz}
    \begin{puyotikz}[\puyobigscale]
        \puyoboard[nrows=6, ncols=3, nhidrows=0]{ybpb/yb/yb///}{}
    \end{puyotikz}
    \begin{puyotikz}[\puyobigscale]
        \puyoboard[nrows=6, ncols=3, nhidrows=0]{ybpb/ybp/ybp///}{}
    \end{puyotikz}
    \begin{puyotikz}[\puyobigscale]
        \puyoboard[nrows=6, ncols=3, nhidrows=0]{ybpby/ybpp/ybp///}{}
    \end{puyotikz}
\end{figure}

What if you received a sequence such as $YB \rightarrow PY \rightarrow BP \rightarrow BY \rightarrow BY \rightarrow PP$? Then you'd have to build the chain like this.

\begin{figure}[H]
    \centering
    \begin{puyotikz}[\puyobigscale]
        \puyoboard[nrows=6, ncols=3, nhidrows=0]{yb/////}{}
    \end{puyotikz}
    \begin{puyotikz}[\puyobigscale]
        \puyoboard[nrows=6, ncols=3, nhidrows=0]{ybp/y////}{}
    \end{puyotikz}
    \begin{puyotikz}[\puyobigscale]
        \puyoboard[nrows=6, ncols=3, nhidrows=0]{ybp/ybp////}{}
    \end{puyotikz}
    \begin{puyotikz}[\puyobigscale]
        \puyoboard[nrows=6, ncols=3, nhidrows=0]{ybp/ybp/yb///}{}
    \end{puyotikz}
    \begin{puyotikz}[\puyobigscale]
        \puyoboard[nrows=6, ncols=3, nhidrows=0]{ybpby/ybp/yb///}{}
    \end{puyotikz}
    \begin{puyotikz}[\puyobigscale]
        \puyoboard[nrows=6, ncols=3, nhidrows=0]{ybpby/ybpp/ybp///}{}
    \end{puyotikz}
\end{figure}

Looks awkward? That's because it is. Once we start discussing ways to build reliable chains, you'll realize that there are certain sequences of pieces that fit comfortably into your chain; and then there are sequences that you struggle to put together. Generally, you want to avoid the latter. If a sequence doesn't work well for a certain chain pattern, it'll probably be good for a different pattern. Flexibility is key.

\subsubsection{Roadblock 4: Garbage}

If your opponent gets enough score very early on in the match, they can use it to pop a quick 1- or 2-chain that sends very little garbage to your board. Depending on where the garbage falls (which depends on RNG!) your life might become hell all of a sudden. Look at these examples of what could happen.

\begin{figure}[H]
    \centering
    \begin{puyotikz}[\puyobigscale]
        \puyoboard[nrows=6, ncols=3, nhidrows=0]{ybp/yb/nb///}{}
    \end{puyotikz}
    \begin{puyotikz}[\puyobigscale]
        \puyoboard[nrows=6, ncols=3, nhidrows=0]{ybp/yn/yb///}{}
    \end{puyotikz}
    \begin{puyotikz}[\puyobigscale]
        \puyoboard[nrows=6, ncols=3, nhidrows=0]{ybn/ybn/ybn///}{}
    \end{puyotikz}
    \begin{puyotikz}[\puyobigscale]
        \puyoboard[nrows=6, ncols=3, nhidrows=0]{nbpb/ybpp/ybp///}{}
    \end{puyotikz}
\end{figure}

All of these examples force you to deviate from your original chaining plan. Of course, you can always try to clear the garbage to rebuild the chain you envisioned at the start, but that takes too much time, and it only gets worse thanks to the other roadblocks. Thankfully, this is a rather rare roadblock; but certain opponents exploit this rarity really well!

\subsection{Conclusion}

If you made it this far, you now understand that Puyo is a puzzle fighting game with simple, yet challenging mechanics, and plenty of things to consider, even at the most basic level. Most importantly, you're now aware that it's definitely not Candy Crush.

If this introduction felt like information overload, don't worry! Take it easy. My goal here was to introduce a bunch of concepts to you, and some of them are hard to grasp at first. The next chapters will focus on specific topics, and hopefully you won't find my logic hard to follow.